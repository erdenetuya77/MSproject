% Options for packages loaded elsewhere
% Options for packages loaded elsewhere
\PassOptionsToPackage{unicode}{hyperref}
\PassOptionsToPackage{hyphens}{url}
\PassOptionsToPackage{dvipsnames,svgnames,x11names}{xcolor}
%
\documentclass[
  a4paperpaper,
  DIV=11,
  numbers=noendperiod]{scrartcl}
\usepackage{xcolor}
\usepackage[left=2cm,right=2cm,top=2cm,bottom=3cm]{geometry}
\usepackage{amsmath,amssymb}
\setcounter{secnumdepth}{5}
\usepackage{iftex}
\ifPDFTeX
  \usepackage[T1]{fontenc}
  \usepackage[utf8]{inputenc}
  \usepackage{textcomp} % provide euro and other symbols
\else % if luatex or xetex
  \usepackage{unicode-math} % this also loads fontspec
  \defaultfontfeatures{Scale=MatchLowercase}
  \defaultfontfeatures[\rmfamily]{Ligatures=TeX,Scale=1}
\fi
\usepackage{lmodern}
\ifPDFTeX\else
  % xetex/luatex font selection
\fi
% Use upquote if available, for straight quotes in verbatim environments
\IfFileExists{upquote.sty}{\usepackage{upquote}}{}
\IfFileExists{microtype.sty}{% use microtype if available
  \usepackage[]{microtype}
  \UseMicrotypeSet[protrusion]{basicmath} % disable protrusion for tt fonts
}{}
\makeatletter
\@ifundefined{KOMAClassName}{% if non-KOMA class
  \IfFileExists{parskip.sty}{%
    \usepackage{parskip}
  }{% else
    \setlength{\parindent}{0pt}
    \setlength{\parskip}{6pt plus 2pt minus 1pt}}
}{% if KOMA class
  \KOMAoptions{parskip=half}}
\makeatother
% Make \paragraph and \subparagraph free-standing
\makeatletter
\ifx\paragraph\undefined\else
  \let\oldparagraph\paragraph
  \renewcommand{\paragraph}{
    \@ifstar
      \xxxParagraphStar
      \xxxParagraphNoStar
  }
  \newcommand{\xxxParagraphStar}[1]{\oldparagraph*{#1}\mbox{}}
  \newcommand{\xxxParagraphNoStar}[1]{\oldparagraph{#1}\mbox{}}
\fi
\ifx\subparagraph\undefined\else
  \let\oldsubparagraph\subparagraph
  \renewcommand{\subparagraph}{
    \@ifstar
      \xxxSubParagraphStar
      \xxxSubParagraphNoStar
  }
  \newcommand{\xxxSubParagraphStar}[1]{\oldsubparagraph*{#1}\mbox{}}
  \newcommand{\xxxSubParagraphNoStar}[1]{\oldsubparagraph{#1}\mbox{}}
\fi
\makeatother


\usepackage{longtable,booktabs,array}
\usepackage{calc} % for calculating minipage widths
% Correct order of tables after \paragraph or \subparagraph
\usepackage{etoolbox}
\makeatletter
\patchcmd\longtable{\par}{\if@noskipsec\mbox{}\fi\par}{}{}
\makeatother
% Allow footnotes in longtable head/foot
\IfFileExists{footnotehyper.sty}{\usepackage{footnotehyper}}{\usepackage{footnote}}
\makesavenoteenv{longtable}
\usepackage{graphicx}
\makeatletter
\newsavebox\pandoc@box
\newcommand*\pandocbounded[1]{% scales image to fit in text height/width
  \sbox\pandoc@box{#1}%
  \Gscale@div\@tempa{\textheight}{\dimexpr\ht\pandoc@box+\dp\pandoc@box\relax}%
  \Gscale@div\@tempb{\linewidth}{\wd\pandoc@box}%
  \ifdim\@tempb\p@<\@tempa\p@\let\@tempa\@tempb\fi% select the smaller of both
  \ifdim\@tempa\p@<\p@\scalebox{\@tempa}{\usebox\pandoc@box}%
  \else\usebox{\pandoc@box}%
  \fi%
}
% Set default figure placement to htbp
\def\fps@figure{htbp}
\makeatother





\setlength{\emergencystretch}{3em} % prevent overfull lines

\providecommand{\tightlist}{%
  \setlength{\itemsep}{0pt}\setlength{\parskip}{0pt}}



 


\usepackage[english,mongolian]{babel}
\usepackage{fontspec}
\setmainfont{Times New Roman}
\setmonofont{DejaVu Sans Mono}
\AddToHook{env/Highlighting/begin}{\footnotesize}
\usepackage{titling}
\pretitle{\begin{center}\LARGE\bfseries}
\posttitle{\par\end{center}\vskip 1em}
\usepackage{titlesec}
\titleformat{\section}{\normalfont\Large\bfseries\selectfont}{\thesection}{1em}{}
\titleformat{\subsection}{\normalfont\large\bfseries\selectfont}{\thesubsection}{1em}{}
\titleformat{\subsubsection}{\normalfont\normalsize\bfseries\selectfont}{\thesubsubsection}{1em}{}
\usepackage{tocloft}
\renewcommand{\cfttoctitlefont}{\Large\bfseries\fontspec{Times New Roman}}
\renewcommand{\cftaftertoctitle}{\vskip 1em}
\renewcommand{\cftsecfont}{\normalfont\selectfont}
\renewcommand{\cftsecpagefont}{\normalfont\selectfont}
\renewcommand{\cftsubsecfont}{\normalfont\selectfont}
\renewcommand{\cftsubsecpagefont}{\normalfont\selectfont}
\renewcommand{\contentsname}{Агуулга}
\KOMAoption{captions}{tableheading,figureheading}
\makeatletter
\@ifpackageloaded{caption}{}{\usepackage{caption}}
\AtBeginDocument{%
\ifdefined\contentsname
  \renewcommand*\contentsname{Table of contents}
\else
  \newcommand\contentsname{Table of contents}
\fi
\ifdefined\listfigurename
  \renewcommand*\listfigurename{List of Figures}
\else
  \newcommand\listfigurename{List of Figures}
\fi
\ifdefined\listtablename
  \renewcommand*\listtablename{List of Tables}
\else
  \newcommand\listtablename{List of Tables}
\fi
\ifdefined\figurename
  \renewcommand*\figurename{Зураг}
\else
  \newcommand\figurename{Зураг}
\fi
\ifdefined\tablename
  \renewcommand*\tablename{Хүснэгт}
\else
  \newcommand\tablename{Хүснэгт}
\fi
}
\@ifpackageloaded{float}{}{\usepackage{float}}
\floatstyle{ruled}
\@ifundefined{c@chapter}{\newfloat{codelisting}{h}{lop}}{\newfloat{codelisting}{h}{lop}[chapter]}
\floatname{codelisting}{Listing}
\newcommand*\listoflistings{\listof{codelisting}{List of Listings}}
\makeatother
\makeatletter
\makeatother
\makeatletter
\@ifpackageloaded{caption}{}{\usepackage{caption}}
\@ifpackageloaded{subcaption}{}{\usepackage{subcaption}}
\makeatother
\usepackage{bookmark}
\IfFileExists{xurl.sty}{\usepackage{xurl}}{} % add URL line breaks if available
\urlstyle{same}
\hypersetup{
  pdftitle={Зээл эргэн төлөх эрсдэлийн үнэлгээ: Naive Bayes ба Decision Tree},
  pdfauthor={Erdenetuya},
  colorlinks=true,
  linkcolor={blue},
  filecolor={Maroon},
  citecolor={Blue},
  urlcolor={Blue},
  pdfcreator={LaTeX via pandoc}}


\title{Зээл эргэн төлөх эрсдэлийн үнэлгээ: Naive Bayes ба Decision Tree}
\author{Erdenetuya}
\date{2025 оны 12-р сарын 5}
\begin{document}
\maketitle
\begin{abstract}
Энэхүү тайланд зээлийн мэдээлэлд тулгуурлан зээл хүсэгчид зээлээ буцаан
төлөх эсэхийг Naive Bayes болон Decision Tree ангилагч ашиглан урьдчилан
таамаглав. Өгөгдөл дээр урьдчилсан боловсруулалт, судалгааны шинжилгээ
(EDA), хоёр төрлийн ангилагчийн гүйцэтгэлийг харьцуулан авч үзэв.
\end{abstract}

\renewcommand*\contentsname{Агуулга}
{
\hypersetup{linkcolor=}
\setcounter{tocdepth}{3}
\tableofcontents
}

\#\textbar{} label: setup \#\textbar{} echo: false

import os import numpy as np import pandas as pd

from sklearn.model\_selection import train\_test\_split from
sklearn.preprocessing import StandardScaler from sklearn.metrics import
( accuracy\_score, precision\_score, recall\_score, f1\_score,
confusion\_matrix, roc\_curve, roc\_auc\_score )

from sklearn.naive\_bayes import GaussianNB from sklearn.tree import
DecisionTreeClassifier

import matplotlib.pyplot as plt

output\_dir = ``\_files'' if not os.path.exists(output\_dir):
os.makedirs(output\_dir)

\section{Өгөгдөл оруулах
жишээ}\label{ux4e9ux433ux4e9ux433ux434ux4e9ux43b-ux43eux440ux443ux443ux43bux430ux445-ux436ux438ux448ux44dux44d}

\section{data =
pd.read\_csv(``project/credit\_data.csv'')}\label{data-pd.read_csvprojectcredit_data.csv}

\section{X = data.drop(``Default'',
axis=1)}\label{x-data.dropdefault-axis1}

\section{y = data{[}``Default''{]}}\label{y-datadefault}

\section{X\_train, X\_test, y\_train, y\_test = train\_test\_split(X, y,
test\_size=0.2,
random\_state=42)}\label{x_train-x_test-y_train-y_test-train_test_splitx-y-test_size0.2-random_state42}

\section{Scaling (хэрэгтэй
бол)}\label{scaling-ux445ux44dux440ux44dux433ux442ux44dux439-ux431ux43eux43b}

\section{scaler = StandardScaler()}\label{scaler-standardscaler}

\section{X\_train =
scaler.fit\_transform(X\_train)}\label{x_train-scaler.fit_transformx_train}

\section{X\_test =
scaler.transform(X\_test)}\label{x_test-scaler.transformx_test}

\#\textbar{} label: naive-bayes \#\textbar{} echo: true

\section{Naive Bayes
сургах}\label{naive-bayes-ux441ux443ux440ux433ux430ux445}

nb\_model = GaussianNB() nb\_model.fit(X\_train, y\_train)

y\_pred\_nb = nb\_model.predict(X\_test) y\_proba\_nb =
nb\_model.predict\_proba(X\_test){[}:,1{]}

metrics\_nb = \{ ``Accuracy'': accuracy\_score(y\_test, y\_pred\_nb),
``Precision'': precision\_score(y\_test, y\_pred\_nb), ``Recall'':
recall\_score(y\_test, y\_pred\_nb), ``F1-score'': f1\_score(y\_test,
y\_pred\_nb), ``AUC'': roc\_auc\_score(y\_test, y\_proba\_nb) \}

metrics\_nb

\#\textbar{} label: decision-tree \#\textbar{} echo: true

\section{Decision Tree
сургах}\label{decision-tree-ux441ux443ux440ux433ux430ux445}

dt\_model = DecisionTreeClassifier(max\_depth=5, random\_state=42)
dt\_model.fit(X\_train, y\_train)

y\_pred\_dt = dt\_model.predict(X\_test) y\_proba\_dt =
dt\_model.predict\_proba(X\_test){[}:,1{]}

metrics\_dt = \{ ``Accuracy'': accuracy\_score(y\_test, y\_pred\_dt),
``Precision'': precision\_score(y\_test, y\_pred\_dt), ``Recall'':
recall\_score(y\_test, y\_pred\_dt), ``F1-score'': f1\_score(y\_test,
y\_pred\_dt), ``AUC'': roc\_auc\_score(y\_test, y\_proba\_dt) \}

metrics\_dt

\#\textbar{} label: compare-metrics \#\textbar{} echo: true \#\textbar{}
tbl-cap: Naive Bayes ба Decision Tree моделийн гүйцэтгэлийн харьцуулалт

compare\_df = pd.DataFrame(\{ ``Naive Bayes'': metrics\_nb, ``Decision
Tree'': metrics\_dt \}) compare\_df

\#\textbar{} label: fig-roc \#\textbar{} fig-cap: Naive Bayes болон
Decision Tree моделуудын ROC муруй \#\textbar{} echo: false

fpr\_nb, tpr\_nb, \_ = roc\_curve(y\_test, y\_proba\_nb) fpr\_dt,
tpr\_dt, \_ = roc\_curve(y\_test, y\_proba\_dt)

fig, ax = plt.subplots(figsize=(6,4)) ax.plot(fpr\_nb, tpr\_nb,
label=f''Naive Bayes (AUC=\{metrics\_nb{[}`AUC'{]}:.2f\})``)
ax.plot(fpr\_dt, tpr\_dt, label=f''Decision Tree
(AUC=\{metrics\_dt{[}`AUC'{]}:.2f\})``) ax.plot({[}0,1{]},{[}0,1{]},
linestyle=''--``, color=''grey'') ax.set\_xlabel(``False Positive
Rate'') ax.set\_ylabel(``True Positive Rate'') ax.legend()
fig.tight\_layout()

fig\_path = os.path.join(output\_dir, ``roc\_compare.pdf'')
fig.savefig(fig\_path) plt.close(fig)




\end{document}
